%\iffalse
\let\negmedspace\undefined
\let\negthickspace\undefined
\documentclass[journal,12pt,onecolumn]{IEEEtran}
\usepackage{cite}
\usepackage{amsmath,amssymb,amsfonts,amsthm}
%\usepackage{algorithmic}
\usepackage{graphicx}
\usepackage{textcomp}
\usepackage{array}
\usepackage{xcolor}
\usepackage{txfonts}
\usepackage{listings}
\usepackage{enumitem}
\usepackage{mathtools}
\usepackage{gensymb}
\usepackage[breaklinks=true]{hyperref}
\usepackage{tkz-euclide} % loads  TikZ and tkz-base
\usepackage{listings}
\usepackage{float}
\usepackage{bm}



\newtheorem{theorem}{Theorem}[section]
\newtheorem{problem}{Problem}
\newtheorem{proposition}{Proposition}[section]
\newtheorem{lemma}{Lemma}[section]
\newtheorem{corollary}[theorem]{Corollary}
\newtheorem{example}{Example}[section]
\newtheorem{definition}[problem]{Definition}
%\newtheorem{thm}{Theorem}[section] 
%\newtheorem{defn}[thm]{Definition}
%\newtheorem{algorithm}{Algorithm}[section]
%\newtheorem{cor}{Corollary}
\newcommand{\BEQA}{\begin{eqnarray}}
\newcommand{\EEQA}{\end{eqnarray}}
\newcommand{\define}{\stackrel{\triangle}{=}}
\theoremstyle{remark}
\newtheorem{rem}{Remark}
%\bibliographystyle{ieeetr}
\begin{document}
%
\providecommand{\pr}[1]{\ensuremath{\Pr\left(#1\right)}}
\providecommand{\prt}[2]{\ensuremath{p_{#1}^{\left(#2\right)} }}        % own macro for this question
\providecommand{\qfunc}[1]{\ensuremath{Q\left(#1\right)}}
\providecommand{\sbrak}[1]{\ensuremath{{}\left[#1\right]}}
\providecommand{\lsbrak}[1]{\ensuremath{{}\left[#1\right.}}
\providecommand{\rsbrak}[1]{\ensuremath{{}\left.#1\right]}}
\providecommand{\brak}[1]{\ensuremath{\left(#1\right)}}
\providecommand{\lbrak}[1]{\ensuremath{\left(#1\right.}}
\providecommand{\rbrak}[1]{\ensuremath{\left.#1\right)}}
\providecommand{\cbrak}[1]{\ensuremath{\left\{#1\right\}}}
\providecommand{\lcbrak}[1]{\ensuremath{\left\{#1\right.}}
\providecommand{\rcbrak}[1]{\ensuremath{\left.#1\right\}}}
\newcommand{\sgn}{\mathop{\mathrm{sgn}}}
\providecommand{\abs}[1]{\left\vert#1\right\vert}
\providecommand{\res}[1]{\Res\displaylimits_{#1}} 
\providecommand{\norm}[1]{\left\lVert#1\right\rVert}
%\providecommand{\norm}[1]{\lVert#1\rVert}
\providecommand{\mtx}[1]{\mathbf{#1}}
\providecommand{\mean}[1]{E\left[ #1 \right]}
\providecommand{\cond}[2]{#1\middle|#2}
\providecommand{\fourier}{\overset{\mathcal{F}}{ \rightleftharpoons}}
\newenvironment{amatrix}[1]{%
  \left(\begin{array}{@{}*{#1}{c}|c@{}}
}{%
  \end{array}\right)
}
%\providecommand{\hilbert}{\overset{\mathcal{H}}{ \rightleftharpoons}}
%\providecommand{\system}{\overset{\mathcal{H}}{ \longleftrightarrow}}
	%\newcommand{\solution}[2]{\textbf{Solution:}{#1}}
\newcommand{\solution}{\noindent \textbf{Solution: }}
\newcommand{\cosec}{\,\text{cosec}\,}
\providecommand{\dec}[2]{\ensuremath{\overset{#1}{\underset{#2}{\gtrless}}}}
\newcommand{\myvec}[1]{\ensuremath{\begin{pmatrix}#1\end{pmatrix}}}
\newcommand{\mydet}[1]{\ensuremath{\begin{vmatrix}#1\end{vmatrix}}}
\newcommand{\myaugvec}[2]{\ensuremath{\begin{amatrix}{#1}#2\end{amatrix}}}
\providecommand{\rank}{\text{rank}}
\providecommand{\pr}[1]{\ensuremath{\Pr\left(#1\right)}}
\providecommand{\qfunc}[1]{\ensuremath{Q\left(#1\right)}}
	\newcommand*{\permcomb}[4][0mu]{{{}^{#3}\mkern#1#2_{#4}}}
\newcommand*{\perm}[1][-3mu]{\permcomb[#1]{P}}
\newcommand*{\comb}[1][-1mu]{\permcomb[#1]{C}}
\providecommand{\qfunc}[1]{\ensuremath{Q\left(#1\right)}}
\providecommand{\gauss}[2]{\mathcal{N}\ensuremath{\left(#1,#2\right)}}
\providecommand{\diff}[2]{\ensuremath{\frac{d{#1}}{d{#2}}}}
\providecommand{\myceil}[1]{\left \lceil #1 \right \rceil }
\newcommand\figref{Fig.~\ref}
\newcommand\tabref{Table~\ref}
\newcommand{\sinc}{\,\text{sinc}\,}
\newcommand{\rect}{\,\text{rect}\,}
%%
%	%\newcommand{\solution}[2]{\textbf{Solution:}{#1}}
%\newcommand{\solution}{\noindent \textbf{Solution: }}
%\newcommand{\cosec}{\,\text{cosec}\,}
%\numberwithin{equation}{section}
%\numberwithin{equation}{subsection}
%\numberwithin{problem}{section}
%\numberwithin{definition}{section}
%\makeatletter
%\@addtoreset{figure}{problem}
%\makeatother

%\let\StandardTheFigure\thefigure
\let\vec\mathbf

\bibliographystyle{IEEEtran}





\bigskip

%\renewcommand{\thefigure}{\theenumi}
%\renewcommand{\thetable}{\theenumi}
%\renewcommand{\theequation}{\theenumi}

Q:  The block diagram of a two-tap high-pass FIR filter is shown below. The filter transfer function is given by $H(z) = Y(z)/X(z)$.\\
If the ratio of maximum to minimum value of $H(z)$ is 2 and $\abs{H(z)}_{max} = 1$, the coefficients $\beta_0$ and $\beta_1$ are \underline{\hspace{3cm}} and \underline{\hspace{3cm}}, respectively. 

\begin{figure}[H]
    \centering
    \includegraphics[width=0.5\linewidth]{figs/qfig.png} 
    \caption{Block diagram}
    \label{fig:GATE23BM39.1}
\end{figure}


\begin{enumerate}[label=(\Alph*)]
\item 0.75, -0.25
\item 0.67, 0.33
\item 0.60, -0.40
\item -0.64, 0.36
\end{enumerate}
\hfill{GATE BM 2022}


\solution \\

\begin{table}[!ht]
    \centering
        \begin{tabular}{|c|c|c|} 
      \hline
\textbf{Variable}& \textbf{Description}& \textbf{Value}\\\hline
	 $\bm{x}(t)$ & state variable & - \\\hline
         $\dot{\bm{x}}(t)$ & derivative of x(t) w.r.t t & $\frac{d\bm{x}(t)}{dt}$\\\hline  
         A & $2x2$ matrix & -\\\hline
         $\bm{x}(0)$ & initial condition & $\sum_{i=1}^{2}\alpha _i\lambda _i$ \\\hline
         $\lambda _i$ for $i = 1, 2$ & eigen values of A & - \\\hline
         $v_i$ for $i = 1, 2$ & eigen vectors of A & - \\\hline
         $\alpha _i$ for $i = 1, 2$ & constants for $\bm{x}(0)$ i.e. initial conditions & - \\\hline
    \end{tabular}

    \caption{input parameters}
    \label{tab:GATE22BM39.1}
\end{table}
As z lies on a unit circle (in argand plane),
\begin{align}
\abs{z} &= 1 \\\implies
\abs{z^{-1}} &= 1 \label {eq:GATE22BM39.1}
\end{align}
Since $H(z)$ is complex, on using Triangle Inequality, we get
\begin{align}
\abs{x + y} \leq \abs{x} + \abs{y} 
\end{align}
And its corollary
\begin{align}
\abs{\abs{x}-\abs{y}} \leq \abs{x + y}
\end{align}
where x and y are complex numbers.
\begin{align}
\abs{\abs{z^{-1}\beta_0}-\abs{\beta_1}} \leq \abs{z^{-1}\beta_0 + \beta_1} \leq \abs{z^{-1}\beta_0} + \abs{\beta_1} 
\end{align}
From \tabref{tab:GATE22BM39.1}
\begin{align}
\abs{\abs{z^{-1}\beta_0}-\abs{\beta_1}} \leq \abs{H(z)} \leq \abs{z^{-1}\beta_0} + \abs{\beta_1}
\end{align}
From \eqref{eq:GATE22BM39.1} 
\begin{align}
\abs{\abs{\beta_0}-\abs{\beta_1}} \leq \abs{H(z)} \leq \abs{\beta_0} + \abs{\beta_1} 
\end{align}
Indivisually, we have
\begin{align}
\abs{H(z)}_{max} \leq \abs{\beta_0} + \abs{\beta_1} 
\end{align}
Now from \tabref{tab:GATE22BM39.1}
\begin{align}
1 \leq \abs{\beta_0} + \abs{\beta_1} \label {eq:GATE22BM39.2}
\end{align}
Similarly,
\begin{align}
\frac{1}{2} \geq \abs{\abs{\beta_0}-\abs{\beta_1}} \label {eq:GATE22BM39.3}
\end{align}
When we verify the options, only one options satisfy both \eqref{eq:GATE22BM39.2} and \eqref{eq:GATE22BM39.3}, i.e. option (A) \\
Infact it satisfies the equality in corollary of triangle inequality.




\end{document}
