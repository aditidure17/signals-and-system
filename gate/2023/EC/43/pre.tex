\documentclass{beamer}
\usetheme{Madrid}

\usepackage{amsmath, amssymb, amsthm}
\usepackage{graphicx}
\usepackage{listings}
\usepackage{gensymb}
\usepackage[utf8]{inputenc}
\usepackage{hyperref}
\usepackage{gvv}

\providecommand{\mtx}[1]{\mathbf{#1}}

\begin{document}

\title{GATE 23 EC }
\author{EE23BTECH11016 - Aditi Dure$^{*}$}
\date{}
\frame{\titlepage}

\begin{frame}
\frametitle{Question 43}
%content
Q: The state equation of a second order system is \\
$ \dot{{x}}(t) = A{x}(t)$, \quad ${x}(0)$ is the initial condition. \\
Suppose $\lambda_1$ and $\lambda_2$ are two distinct eigenvalues of $A$, and $\nu_1$ and $\nu_2$ are the corresponding eigenvectors. For constants $\alpha_1$ and $\alpha_2$, the solution, ${x}(t)$, of the state equation is \\
\begin{enumerate}
\item $\sum_{i=1}^{2} \alpha_ie^{\lambda_it}\bf{\nu}_i$
\item $\sum_{i=1}^{2} \alpha_ie^{2\lambda_it}\bf{\nu}_i$
\item $\sum_{i=1}^{2} \alpha_ie^{3\lambda_it}\bf{\nu}_i$
\item $\sum_{i=1}^{2} \alpha_ie^{4\lambda_it}\bf{\nu}_i$
\end{enumerate}
\end{frame}

\begin{frame}
\frametitle{Input Parameter Table}
%content
\begin{table}[!ht]
    \centering
        \begin{tabular}{|c|c|c|} 
      \hline
\textbf{Variable}& \textbf{Description}& \textbf{Value}\\\hline
	 $H(z)$ & Transfer Function & $\beta_0z^{-1} + \beta_1$ \\\hline
         $\abs{H(z)}_{max}$ & Maximum value of Transfer Function & 1 \\\hline  
         $\abs{H(z)}_{min}$ & Minimum value of Transfer Function & $\frac{1}{2}$\\\hline
    \end{tabular}

    \caption{input parameters}
    \label{tab:gate23EC43.1}
\end{table}
\end{frame}

\begin{frame}
\frametitle{Proofs}
%content
\begin{align}
A\vec{y} &= \lambda \vec{y} 
\end{align}
Eigen values of inverse of a matrix is reciprocal of eigen value of the given matrix\\
\begin{align}
A^{-1}A\vec{y} &= A^{-1}\lambda \vec{y} \\ \implies
\vec{y} &= \lambda A^{-1}\vec{y}  \\ \implies
A^{-1} \vec{y} &= \frac{1}{\lambda}\vec{y} \label{eq:gate23EC43.1}
\end{align}
\end{frame}


\begin{frame}
\frametitle{Proofs - Continued}
%content
Eigen value of a matrix shifts by the same amount as that of the matrix.\\
\begin{align}
(A - \sigma I)\vec{y} &= A\vec{y} - \sigma I\vec{y} \\
&= \lambda \vec{y} - \sigma \vec{y} \\
&= (\lambda - \sigma) \vec{y} \label{eq:gate23EC43.2}
\end{align}
\end{frame}

\begin{frame}
\frametitle{Solution}
%content
Using Laplace transform: \\
Given Equation:
\begin{align}
\dot{{x}}(t) &= A{x}(t) \\
\frac{d{x}(t)}{dt} &= A{x}(t) 
\end{align}
\end{frame}

\begin{frame}
\frametitle{Solution - Continued}
%content
Taking Laplace Transform:
\begin{align}
\mathcal{L}\brak{\frac{d{x}(t)}{dt}} &= \mathcal{L}\brak{A{x}(t)} \\
sX(s) - {x}(0) &= AX(s) \\
(sI - A)X(s) &= x(0) \\
X(s) &= (sI - A)^{-1}{x}(0) \label{eq:gate23EC43.3}
\end{align}
\end{frame}

\begin{frame}
\frametitle{Solution - Continued}
%content
From \tabref{tab:gate23EC43.1}, we can write $x(0)$ in terms of two linearly independent variables as 
\begin{align}
    x(0) &= \alpha_1v_1 + \alpha_2v_2 \\
    &= \sum_{i=1}^{2}\alpha_iv_i \label{eq:gate23EC43.4}
\end{align}
From \eqref{eq:gate23EC43.3}, \eqref{eq:gate23EC43.4}

\begin{align}
 X(s) &= (sI - A)^{-1}\brak{\sum_{i=1}^{2}\alpha_iv_i} \\
 &= \sum_{i=1}^{2}(sI - A)^{-1}\alpha_iv_i
\end{align}
\end{frame}

\begin{frame}
\frametitle{Solution - Continued}
%content
From \eqref{eq:gate23EC43.1}, \eqref{eq:gate23EC43.2}
\begin{align}
X(s) &=  \sum_{i=1}^{2}\frac{1}{s-\lambda_i}\brak{\alpha_iv_i} 
\end{align}
Now take inverse Laplace Transform
\begin{align}
\mathcal{L}^{-1}\brak{X(s)} &= \mathcal{L}^{-1}\brak{\sum_{i=1}^{2}\frac{1}{s-\lambda_i}\brak{\alpha_iv_i}} \\
{x}(t) &= \sum_{i=1}^{2} e^{\lambda_it}\brak{\alpha_iv_i} 
\end{align}
\end{frame}




\end{document}
